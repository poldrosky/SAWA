 
\chapter{Proceso de Instalación}

Para poder instalar la aplicación debemos instalar PostgreSQL 9.1, Glassfish 3.1.2.2 y la extensión
para postgresql llamada pg\_similarity, la aplicación solo puede ser instalada bajo sistemas
operativos GNU/Linux


Para la instalación de PostgreSQL 9.1 (Para Debian/Linux), en una terminal se ejecuta los siguiente:\\

\lstset{backgroundcolor=\color{white},frame=shadowbox, basicstyle=\small,  breaklines=true, breaklines=true,language=bash}
\begin{lstlisting}
$ su
# apt-get install postgresql postgresql-client pgadmin3
postgresql-server-dev-all postgresql-contrib
\end{lstlisting}

Ahora se borra la contraseña para la cuenta de  administrador de ``postgres'' para ello se ejecuta lo
siguiente en la linea de comandos. \\

\lstset{backgroundcolor=\color{white},frame=shadowbox, basicstyle=\small,  breaklines=true, breaklines=true,language=bash}
\begin{lstlisting}
# su postgres -c psql template1
\end{lstlisting}

\lstset{backgroundcolor=\color{white},frame=shadowbox, basicstyle=\small,  breaklines=true, breaklines=true,language=sql}
\begin{lstlisting}
template1=# ALTER USER postgres WITH PASSWORD `postgres1';
template1=#\q
\end{lstlisting}

Eso altera la contraseña dentro de la base de datos, ahora se tiene que hacer lo mismo para el  
usuario 'postgres' y colocar la misma contraseña que utilizó anteriormente. \\

\lstset{backgroundcolor=\color{white},frame=shadowbox, basicstyle=\small,  breaklines=true, breaklines=true,language=bash}
\begin{lstlisting}
# passwd -d postgres
 #su postgres -c passwd
\end{lstlisting}

Ahora para instalar pg\_similarity primero se descarga desde github de la siguiente manera

\lstset{backgroundcolor=\color{white},frame=shadowbox, basicstyle=\small,  breaklines=true, breaklines=true,language=bash}
\begin{lstlisting}
$ git clone https://github.com/eulerto/pg_similarity.git 
\end{lstlisting}

Luego para que pueda ser usada la extensión se la compila como superusuario de la siguiente manera.

\lstset{backgroundcolor=\color{white},frame=shadowbox, basicstyle=\small,  breaklines=true, breaklines=true,language=bash}
\begin{lstlisting}
# su
# cd pg_similarity
# USE_PGXS=1 make
# USE_PGXS=1 make install
# su postgres
$ createdb sawa
\end{lstlisting}

Luego se crea las funciones y se restaura la copia de la base de datos sawa.sql

\lstset{backgroundcolor=\color{white},frame=shadowbox, basicstyle=\small,  breaklines=true, breaklines=true,language=bash}
\begin{lstlisting}
$ psql sawa
sawa=# create extension pg_similarity;
sawa=# \i sawa.sql
\end{lstlisting}

\textbf{Instalación de Glassfish}

Primero se descarga la versión 3.1.2.2 para GNU/linux desde la página de Oracle\footnote{http://www.oracle.com/technetwork/middleware/glassfish/downloads/ogs-3-1-1-downloads-439803.html}
y se la ejecuta.

\lstset{backgroundcolor=\color{white},frame=shadowbox, basicstyle=\small,  breaklines=true, breaklines=true,language=bash}
\begin{lstlisting}
$ sh ogs-3.1.2.2-unix-ml.sh
\end{lstlisting}

Hay que descargar el driver JDBC de postgresql desde la 
página de PostgreSQL\footnote{http://jdbc.postgresql.org} y copiarlo en 
el directorio glassfish3/glassfish/domains/domain1/lib. Para iniciar el servidor 
se ejecuta lo siguiente


\lstset{backgroundcolor=\color{white},frame=shadowbox, basicstyle=\small,  breaklines=true, breaklines=true,language=bash}
\begin{lstlisting}
$ ./glassfish3/glassfish/bin/startserv
\end{lstlisting}

Con esto ya en en el navegador se inglesa con la dirección http://localhost:8080, y
entramos a la consola de administración, ingresamos usuario y contraseña de haber escrito una
en la instalación de glassfish.

Ir  a ``Resources/JDBC/Connection Pools'' y crear una nueva conexión con los datos que muestra la
Figura~\ref{figura:m1} y luego clic en siguiente.

\begin{figure}[!ht]
\begin{center}
\includegraphics[width=13cm]{pictures/m1.png}
\end{center}
\caption{Crear Nueva conexión} \label{figura:m1}
\end{figure}

Seleccione el origen de datos de nombre de clase org.postgresql.ds.PGConnectionPoolDataSource 
y escribir a las siguientes propiedades adicionales como muestra la Figura~\ref{figura:m2}.

\begin{figure}[!ht]
\begin{center}
\includegraphics[width=13cm]{pictures/m2.png}
\end{center}
\caption{Propiedades adicionales de conexión} \label{figura:m2}
\end{figure}

Ya con esto guardamos las conexiones y damos clic en finalizar para guardar la conexión.

Luego en ``Resources/JDBC/JDBC Resources'' escribimos en nombre JNDI y escogemos el Pool Name creado 
anteriormente como lo muestra la  Figura~\ref{figura:m3}.

\begin{figure}[!ht]
\begin{center}
\includegraphics[width=13cm]{pictures/m3.png}
\end{center}
\caption{Recursos JDBC} \label{figura:m3}
\end{figure}

Ya por último en ``Applications'' escogemos en donde tenemos almacenada la aplicación web 
que tiene extensión ``.war'' como muestra la  Figura~\ref{figura:m4}.

\begin{figure}[!ht]
\begin{center}
\includegraphics[width=13cm]{pictures/m4.png}
\end{center}
\caption{Subir Aplicación} \label{figura:m4}
\end{figure}

Si todo sale bien nos aparecerá una ventana con la url de la aplicación 
como muestra la Figura~\ref{figura:m5}. y al 
ingresar a la aplicación nos desplegara la ventana de inicio de la aplicación como lo muestra la
Figura~\ref{figura:m6}.

\begin{figure}[!ht]
\begin{center}
\includegraphics[width=10cm]{pictures/m5.png}
\end{center}
\caption{Url de la Aplicación} \label{figura:m5}
\end{figure}

\begin{figure}[!ht]
\begin{center}
\includegraphics[width=13cm]{pictures/m6.png}
\end{center}
\caption{Aplicación} \label{figura:m6}
\end{figure}

\textbf{Nota:} Si no desea instalar la aplicación y quiere ver su funcionamiento la aplicación esta alojada en \url{http://ingenieria.udenar.edu.co:8080/Sawa/}
